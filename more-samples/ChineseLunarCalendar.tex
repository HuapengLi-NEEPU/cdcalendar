% !TEX program=xelatex

% nobabel means you'll need to provide your own month names, week day headings,
% etc
\documentclass[17pt,nobabel,sundayweek,giant,rgb]{cdcalendar}

%% and here we do some settings for a calendar in Chinese
%% (*not* the lunar calendar! Just localising the Gregorian calendar into
%% Chinese)
\usepackage{zh-mod}
\newcommand{\weibo}[1]{{\xeCJKsetup{CJKecglue={}}\faWeibo{}\,@#1}}

\usepackage[rm]{roboto}

\usepackage{csvsimple,xstring}

\robustify{\\}
\newcommand{\annotdate}[2]{%
  \ifboolexpr{ test{\IfBeginWith{#2}{初}}
            or test{\IfBeginWith{#2}{十}}
            or test{\IfEndWith{#2}{十}}
            or test{\IfBeginWith{#2}{廿}}
  }{\def\extra{}}%
   {\def\extra{正赤}}%
  \node[font=\tiny\sffamily\CJKspace,\extra,
  anchor=north,outer sep=0.55ex,align=flush center]
  at (cdCal-\thepage-#1) {#2};%
}

\newcommand{\importlunar}[1]{%
\begin{scope}[on background layer]%
  %% lunar-2021.csv contains mappings between the Chinese lunar calendar and
  %% 2021 Gregorian calendar, with columns EVENT and DATE.
  %% Sourced from https://github.com/infinet/lunar-calendar/blob/master/chinese_lunar_prev_year_next_year.ics; converted to Simplified Chinese;
  %% converted to .csv with https://www.indigoblue.eu/ics2csv/
  \csvreader[filter test=\IfBeginWith{\LNDATE}{#1}]
  {lunar-2021.csv}{EVENT=\LNEVENT,DATE=\LNDATE}
  {\annotdate{\LNDATE}{\LNEVENT}}
\end{scope}%
}

%%%%%%
% Some settings for the monthly calendars, to accomodate extra annotations below each date
%%%%%%
\dayHeadingStyle{font=\sffamily\color{gray!90}}
\sundayColor{酡红}
\monthTitleStyle{font={\fontsize{42pt}{44pt}\selectfont\fangsong},text=正青}
\eventStyle{\scriptsize\heiti\def\CJKecglue{}}
\renewcommand\printeventname[1]{{\sffamily #1}}
\tikzset{every calendar/.append style={day yshift=2.5\ccwd,day xshift=2.75\ccwd},every day/.append style={font=\bfseries}}
\patchcmd{\monthCalendar}{0pt,\paperheight}{0pt,\dimexpr\paperheight-0.5cm\relax}{}{}

%% 月份上下加的小点缀,嫌太花了的可以把这些删掉。
\usepackage{scrlfile}
\ReplacePackage{xpatch}{regexpatch}
\usepackage{pgfornament-han}
\usepackage{cncolours}
\patchcmd{\monthCalendar}
  {\pgfcalendarmonthname}
  {\rotatebox{90}{\pgfornamenthan[width=.8\ccwd,color=淡青]{43}}\\%
   \pgfcalendarmonthname}
  {}{}
\patchcmd{\monthCalendar}
  {{\rotatebox{-90}{#1}}}
  {{\rotatebox{-90}{#1}}\\[0.75ex]%
    \rotatebox{90}{\pgfornamenthan[width=.8\ccwd,color=淡青]{43}}}
  {}{}

%% Define all event mark styles here
\tikzset{holiday/.style={rectangle,fill=牙色,inner xsep=1.375em,inner ysep=0.5em}}
\tikzset{pink icon/.style={text=水红,font=\Large,yshift=0.2ex}}
\tikzset{blue icon/.style={text=淡青,font=\Large,yshift=0.2ex}}


\usepackage[fixed]{fontawesome5}
\usepackage{graphicx}
\usepackage{wallpaper}
\graphicspath{{img/}}

\setlength{\CalPageMargin}{1cm}
% \setlength{\EventLineWidth}{6in}
\geometry{a4paper}

\begin{document}

%\TileWallPaper{.5\paperwidth}{.5\paperheight}{ricepaper_v3}
\coverImage[\vspace*{-7.25cm}~\weibo{溪畔鹤迹} \url{http://1t.click/bxy3}]{taotie}

\coverTitle[
font=\fontsize{48pt}{50pt}\sffamily\fangsong\bfseries,
text width=\linewidth,align=flush right,正青]
{2021年历(附农历节气)}

\makeCover
\clearpage

%% Uncomment the next line if you want to clear the background
%\ClearWallPaper

\illustration[西汉,错金银青铜锯齿形器。图片拍摄:一只饭包。\weibo{张博力}\url{http://t.cn/EUm09h0}]{\textwidth}{IMG_4645}
% 一、元旦:2021年1月1日至3日放假,共3天。
\begin{monthCalendar}{2021}{01}
  \event[mark style=holiday]{2021-01-01}{2021-01-03}{元旦放假}
  \event{2021-01-11}{2021-01-13}{ACME Conference}
  \event[mark style=pink icon,marker=\faBirthdayCake]{2021-01-23}{}{朋友生日}
  \importlunar{2021-01}
\end{monthCalendar}

\clearpage

\illustration[商 有领玉璧 1986年广汉三星堆遗址K2出土 四川广汉三星堆博物馆藏 \weibo{川后} \url{http://t.cn/EUmH7He}]{\textwidth}{youlingyupi}

% 二、春节:2月11日至17日放假调休,共7天。2月7日(星期日)、2月20日(星期六)上班。
\begin{monthCalendar}{2021}{02}
  \event[mark style=holiday]{2021-02-11}{2021-02-17}{春节放假调休}
  \event[mark style=blue icon,marker=\faBriefcase]{2021-02-07}{}{上班日}
  \event[mark style=blue icon,marker=\faBriefcase]{2021-02-20}{}{上班日}
  \importlunar{2021-02}
\end{monthCalendar}



\clearpage

\illustration[Immortals Riding Dragons: Section of a Tomb Pediment.
Han dynasty (206 B.C.--A.D. 220), 1st century B.C./A.D. Probably from Henan province, China. Art Initiative Chicago. \url{http://t.cn/EUmmVjV}]{\textwidth}{dragonriders}

\begin{monthCalendar}{2021}{03}
  \importlunar{2021-03}
\end{monthCalendar}

\clearpage

\illustration[明代金饰件,杭州桃源岭出土,浙江省博物馆 藏。\weibo{一梦琥珀}\url{http://1t.click/bxyv}]{\textwidth}{toushi}

% 三、清明节:4月3日至5日放假调休,共3天。
\begin{monthCalendar}{2021}{04}
  \event[mark style=holiday]{2021-04-03}{3}{清明节放假调休}
  \event[mark style=blue icon,marker=\faBriefcase]{2021-04-25}{}{上班日}
  \importlunar{2021-04}
\end{monthCalendar}

\clearpage

\illustration[西汉,错金银青铜锯齿形器。图片拍摄:一只饭包。\weibo{张博力}\url{http://t.cn/EUm09h0}]{\textwidth}{IMG_4645}

% 四、劳动节:5月1日至5日放假调休,共5天。4月25日(星期日)、5月8日(星期六)上班。
\begin{monthCalendar}{2021}{05}
  \event[mark style=holiday]{2021-05-01}{5}{劳动节放假调休}
  \event[mark style=blue icon,marker=\faBriefcase]{2021-05-08}{}{上班日}
  \importlunar{2021-05}
\end{monthCalendar}

\clearpage

\illustration[商 有领玉璧 1986年广汉三星堆遗址K2出土 四川广汉三星堆博物馆藏 \weibo{川后} \url{http://t.cn/EUmH7He}]{\textwidth}{youlingyupi}

% 五、端午节:6月12日至14日放假,共3天。
\begin{monthCalendar}{2021}{06}
  \event[mark style=holiday]{2021-06-12}{3}{端午节放假调休}
  \importlunar{2021-06}
\end{monthCalendar}

\clearpage

\illustration[Immortals Riding Dragons: Section of a Tomb Pediment.
Han dynasty (206 B.C.--A.D. 220), 1st century B.C./A.D. Probably from Henan province, China. Art Initiative Chicago. \url{http://t.cn/EUmmVjV}]{\textwidth}{dragonriders}

\begin{monthCalendar}{2021}{07}
  \importlunar{2021-07}
\end{monthCalendar}

\clearpage

\illustration[明代金饰件,杭州桃源岭出土,浙江省博物馆 藏。\weibo{一梦琥珀}\url{http://1t.click/bxyv}]{\textwidth}{toushi}

\begin{monthCalendar}{2021}{08}
  \importlunar{2021-08}
\end{monthCalendar}

\clearpage

\illustration[西汉,错金银青铜锯齿形器。图片拍摄:一只饭包。\weibo{张博力}\url{http://t.cn/EUm09h0}]{\textwidth}{IMG_4645}

% 六、中秋节:9月19日至21日放假调休,共3天。9月18日(星期六)上班。
\begin{monthCalendar}{2021}{09}
  \event[mark style=holiday]{2021-09-19}{3}{中秋节放假调休}
  \event[mark style=blue icon,marker=\faBriefcase]{2021-09-18}{}{上班日}
  \event[mark style=blue icon,marker=\faBriefcase]{2021-09-26}{}{上班日}
  \importlunar{2021-09}
\end{monthCalendar}

\clearpage

\illustration[商 有领玉璧 1986年广汉三星堆遗址K2出土 四川广汉三星堆博物馆藏 \weibo{川后} \url{http://t.cn/EUmH7He}]{\textwidth}{youlingyupi}

% 七、国庆节:10月1日至7日放假调休,共7天。9月26日(星期日)、10月9日(星期六)上班。
\begin{monthCalendar}{2021}{10}
  \event[mark style=holiday]{2021-10-01}{2021-10-07}{国庆节放假调休}
  \event[mark style=blue icon,marker=\faBriefcase]{2021-10-09}{}{上班日}
  \importlunar{2021-10}
\end{monthCalendar}

\clearpage

\illustration[Immortals Riding Dragons: Section of a Tomb Pediment.
Han dynasty (206 B.C.--A.D. 220), 1st century B.C./A.D. Probably from Henan province, China. Art Initiative Chicago. \url{http://t.cn/EUmmVjV}]{\textwidth}{dragonriders}

\begin{monthCalendar}{2021}{11}
  \importlunar{2021-11}
\end{monthCalendar}

\clearpage

\illustration[明代金饰件,杭州桃源岭出土,浙江省博物馆 藏。\weibo{一梦琥珀}\url{http://1t.click/bxyv}]{\textwidth}{toushi}

\begin{monthCalendar}{2021}{12}
  \importlunar{2021-12}
\end{monthCalendar}

\end{document}
